\documentclass[one]{article}
\usepackage[margin=2.54cm]{geometry}
\usepackage{amsthm}
\newtheorem{theorem}{Theorem}
\newtheorem{definition}{Definition}
\begin{document}

\tableofcontents

\section{Idea}

This project is to formalize much of the high school mathematics curriculum in cohesive homotopy type theory. This includes the theory of equations in the real numbers, elementary functions over the real numbers, and analytic Euclidean planar and volumetric geometry, as well as miscellaneous topics such as elementary combinatorics, number theory, probability, and statistics. 

We work in cohesive homotopy type theory rather than plain homotopy type theory, because major topics in high school geometry, such as the perimeter and area of planar geometric figures, requires proving theorems about the topological properties of the real numbers, subtypes of the Euclidean plane, and continuous functions thereof. Due to the synthetic treatment of topological spaces and continuous functions in cohesive homotopy type theory, the proofs in cohesive homotopy type theory tend to be shorter than in plain homotopy type theory: whereas in plain homotopy type theory the theorems need to be appended with statements that types have the proper topology based off the Euclidean metric, and functions are continuous in a suitable way, in cohesive homotopy type theory, all types which are cohesive already have the proper topology, and all functions which are elements of cohesive function types are already continuous. 

\section{Foundations and synthetic topology}

We use agda notation, predicate logic notation [insert article reference to Charles Sanders Pierce], and arithmetic notation interchangeably for dependent pair and dependent function types: 

\begin{table}
\centering
\begin{tabular}{llll}
Notation                & Agda notation                          & Predicate logic notation             & Arithmetic notation                     \\
Interpretation           & Functions/pairs      & For all/there exists    &   Products/sums            \\
Dependent function types & $(x : A) \to B(x)$    & $\Pi x:A.B(x)$    & $\prod_{x:A} B(x)$  \\
Dependent pair types     & $(x : A) \times B(x)$ & $\Sigma x:A.B(x)$ & $\sum_{x:A} B(x)$   \\                 
\end{tabular}
\end{table}

\begin{itemize}
\item dependent type theory introduced as follows: identity types, pair types, function types, dependent pair types, dependent function types, sum types, empty type, unit type, equivalences, and function extensionality. Also, a type of all propositions, which means that the theory is globally impredicative, as well as propositional truncation, which allows for predicate logic to be done, and set truncation. 

\item the theory of univalent Tarski universes, predicativity (using types of locally small propositions instead of the type of all propositions), propositional resizing (local impredicativity), local excluded middle, universe of finite types and arithmetic of finite types, and modalities

\item the Tarski universes $(U_\mathrm{crisp}, \mathrm{El}_\mathrm{crisp})$ representing crisp types and $(U_\mathrm{cohesive}, \mathrm{El}_\mathrm{cohesive})$ representing cohesive types, as well as functions 
$$\mathrm{disc}:U_\mathrm{crisp} \hookrightarrow U_\mathrm{cohesive}$$ 
$$\mathrm{codisc}:U_\mathrm{crisp} \hookrightarrow U_\mathrm{cohesive}$$
$$\mathrm{underlying}:U_\mathrm{cohesive} \to U_\mathrm{crisp}$$ 
$$\mathrm{fundamental}:U_\mathrm{cohesive} \to U_\mathrm{crisp}$$ 
necessary to formalize the modalities 
$$\sharp:U_\mathrm{cohesive} \to U_\mathrm{cohesive}$$
$$\flat:U_\mathrm{crisp} \to U_\mathrm{crisp}$$
$$\mathrm{shape}:U_\mathrm{cohesive} \to U_\mathrm{cohesive}$$

This approach is more similar to the earlier approach of Urs Schreiber and Michael Shulman in [insert article reference], compared to the later approach of Michael Shulman in [insert article reference]. 

\item the abstract unit interval $\mathbf{I}:U_\mathrm{cohesive}$, which we shall assume for the time being to only be a type with two non-equal terms, and the associated axiom $\mathbf{I} \flat$ (basically Shulman's axiom C2 for a family indexed by the unit type). This axiom already implies that $\mathbf{I}$ is compact connected and that it's shape is contractible. 

\end{itemize}

\section{Real numbers}

We follow Peter Freyd [insert article reference] in defining $\mathbf{I}$ to be the terminal interval coalgebra, and then define the real numbers accordingly. 

\section{Elementary functions and the theory of equations}

...

\section{Euclidean geometry}

...

\section{Miscellaneous topics}

...

\end{document}

